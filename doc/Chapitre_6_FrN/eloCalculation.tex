\documentclass[11pt]{article}
\usepackage[utf8]{inputenc}	% Para caracteres en español
\usepackage{amsmath,amsthm,amsfonts,amssymb,amscd}
\usepackage{lastpage}
\usepackage{enumitem}
\usepackage{fancyhdr}
\usepackage{mathrsfs}
\usepackage{wrapfig}
\usepackage{url}
\usepackage{setspace}
\usepackage{calc}
\usepackage{diagbox}
\usepackage{multicol}
\usepackage{cancel}
\usepackage[retainorgcmds]{IEEEtrantools}
\usepackage[margin=3cm]{geometry}
\usepackage{amsmath}
\newlength{\tabcont}
\setlength{\parindent}{0.0in}
\setlength{\parskip}{0.05in}
\usepackage{empheq}
\usepackage{framed}
\usepackage[most]{tcolorbox}
\usepackage{xcolor}
\colorlet{shadecolor}{orange!15}
\parindent 0in
\parskip 12pt
\geometry{margin=1in, headsep=0.25in}
\theoremstyle{definition}
\newtheorem{defn}{Definition}
\newtheorem{reg}{Rule}
\newtheorem{exer}{Exercise}
\newtheorem{note}{Note}
\usepackage{cfr-lm}
\begin{document}
\section{Calculer l'ELO après une partie de Quoridor}
    Supposons deux joueurs de ELO: 
    \begin{center}
        r(1) = 1500 et r(2) = 1753
    \end{center}
    Nous allons devoir calculer le "transformed score" R(n) comme suit:
        
    \begin{equation}
        \begin{split}
            & R(1) = 10^{\frac{r(1)}{400}} = 5623,413 \\
            & R(2) = 10^{\frac{r(2)}{400}} = 24126,815 \\
        \end{split}
    \end{equation}

    On va calculer le "expected score" E(n) pour les joueurs:
    \begin{equation}
        \begin{split}
            & E(1) = R(1) / ( R(1) + R(2)) = 5623,413 / (5623,413 + 24126,815) = 0,189 \\
            & E(2) = R(2) / ( R(1) + R(2)) = 24126,815 / (5623,413 + 24126,815) = 0,8109 \\
        \end{split}
    \end{equation}

    On va definir le actual score pour les 2 joueurs:
    \begin{center}
        S(1) = 1 s'il gagne / 0 s'il perd et viceversa pour le joueur 2
    \end{center}
    
    Calculons l'ELO final r'(n):
    Pour cela nous devons definir K:
    This is called the K-factor and basically a measure of how strong a match will impact the players’ ratings.
    If you set K too low the ratings will hardly be impacted by the matches and very stable ratings (too stable) will occur.
    On the other hand, if you set it too high, the ratings will fluctuate wildly according to the current performance.
     Different organizations use different K-factors, there’s no universally accepted value. In chess the ICC uses a value of K = 32. \\
    Ici, on suppose que le joueur 1 ait gagné
    \begin{equation}
        \begin{split}
           & r'(1) = r(1) + K \cdot (S(1) - E(1)) = 1500 + 32 \cdot (1 - 0,189) = 1525,952 \\
           & r'(2) = r(2) + K \cdot (S(2) - E(2)) = 1753 + 32 \cdot (0-0,8109) = 1727,0512 \\
        \end{split}
    \end{equation}

    Les ELO finaux des deux joueurs seront : \\

        r(1) = 1526, +26 \ r(2) = 1727, -26
    \newpage
    Supposons maintenant la partie se dèroule sentre 4 joueurs:
    r(1) = 1700, r(2) = 1900, r(3) = 1500, r(4) = 1600
    On va calculer les R(n):
        \begin{equation}
            \begin{split}
                & R(1) = 10 ^{\frac{1700}{400}} = 17782,794 \\
                & R(2) = 10^{\frac{1900}{400}} = 56243,1325 \\
                & R(3) = 10^{\frac{1500}{400}} = 5623, 413  \\
                & R(4) = 10^{\frac{1600}{400}} = 10000 \\
            \end{split}
        \end{equation}
        Pour le "expected score" E(n) pour chaque joueur, on va prendre la moyenne ponderée des R(m) des autres joueurs.
        On va proceder comme suit: \\
        Pour deux joueurs n et m: \\
        $\varepsilon = $ r(n) - r(m) 
            $$ \omega_{n, m} =
            \begin{cases}
                1 + \varepsilon \cdot 10^{-3}, & \text{si $\varepsilon \le -100$} \\
                1 - \varepsilon \cdot 10^{-3}, & \text{si $\varepsilon \ge 100$} \\
                1, & \text{si $-100 < \varepsilon < 100$ }\\
            \end{cases} $$
        E(1) sera alors avec J le nombre de joueurs adversaires (3): 
        \begin{equation}
            \begin{split}
                & E(1) = R(1) / [R(1) + (\dfrac{R(2) \cdot \omega_{1, 2} + R(3) \cdot \omega_{1, 3} + R(4) \cdot \omega_{1, 4}}{J})] = 0,394 \\
            \end{split}
        \end{equation}
        E(2), E(3) et E(4) se calculeront comme suit:
        \begin{equation}
            \begin{split}
                & E(2) = R(2) / [R(2) + (\dfrac{R(1) \cdot \omega_{2, 1} + R(3) \cdot \omega_{2, 3} + R(4) \cdot \omega_{2,4} }{J})] = 0,872 \\
                & E(3) = R(3) / [R(3) + (\dfrac{R(1) \cdot \omega_{3, 1} + R(2) \cdot \omega_{3, 2} + R(4) \cdot \omega_{3,4} }{J})] = 0,131 \\
                & E(4) = R(4) / [R(4) + (\dfrac{R(1) \cdot \omega_{4, 1} + R(2) \cdot \omega_{4, 2} + R(3) \cdot \omega_{4,3} }{J})] = 0,232 \\
            \end{split}
        \end{equation}
        S(n) sera comme defini : 1 si le joueur a gagné, 0 s'il a perdu. \\
        
        r'(n) se calculera toujours de la meme manière:
        \begin{equation}
            \begin{split}
                & r'(1) = r(1) + K \cdot (S(1) - E(1)) = 1719 \\
                & r'(2) = r(2) + K \cdot (S(2) - E(2)) = 1872 \\
                & r'(3) = r(3) + K \cdot (S(3) - E(3)) = 1495 \\
                & r'(4) = r(4) + K \cdot (S(4) - E(4)) = 1592 \\
            \end{split}
        \end{equation}

        L'ELO final pour chaque joueur correspondra alors à r'(n)

            Source: \url{https://metinmediamath.wordpress.com/2013/11/27/how-to-calculate-the-elo-rating-including-example/}

\end{document}