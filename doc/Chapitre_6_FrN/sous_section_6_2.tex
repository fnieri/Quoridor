\documentclass[11pt,a4paper,titlepage]{article}

% Language
\usepackage{polyglossia}
%\setdefaultlanguage[frenchpart=false, frenchfootnote=true, frenchitemlabels=true]{french}
\usepackage{numprint}

% Fonts
\usepackage{fontspec}
\setmainfont{Linux Libertine O}
\setsansfont{Linux Biolinum O}

\usepackage{mathtools}
\usepackage{amssymb}
\usepackage{amsfonts}
\usepackage{graphicx}
\usepackage{float}
\usepackage{listings}
\usepackage{hyperref}
\usepackage{attachfile2}

\usepackage{fullpage}
\usepackage[parfill]{parskip}

\usepackage{xcolor}

%-----------------------------------------------------------

\newcommand{\up}[1]{\textsuperscript{#1}}
\newcommand{\down}[1]{\textsubscript{#1}}

\newcommand{\addcode}[3]{
    \begin{figure}[H]
        \centering
        \lstinputlisting[language=#2, caption=\textattachfile{#1}{#1}, label=#3]{#1}
    \end{figure}
}

\newcommand{\addimg}[4]{
    \begin{figure}[H]
        \centering
        \includegraphics[#2]{#1}
    \caption{#3}
    \label{#4}
    \end{figure}
}

%-----------------------------------------------------------

\definecolor{base}{RGB}{250,245,237}
\definecolor{subtle}{RGB}{110,107,135}
\definecolor{red}{RGB}{181,99,122}
\definecolor{gold}{RGB}{235,158,51}
\definecolor{rose}{RGB}{214,130,125}
\definecolor{pine}{RGB}{41,105,130}
\definecolor{foam}{RGB}{87,148,158}
\definecolor{iris}{RGB}{143,122,168}

\hypersetup{
    colorlinks=true,
        allcolors=iris
}
\attachfilesetup{color=iris}

%-----------------------------------------------------------

\lstset{
    % backgroundcolor=\color{white},        % background color
        basicstyle=\ttfamily,                   % regular style
        breakatwhitespace=true,                 % sets if automatic breaks should only happen at whitespace
        breaklines=true,                        % sets automatic line breaking
        captionpos=t,                           % caption-position
        commentstyle=\itshape\color{subtle},    % comment style
        % deletekeywords={...},                 % delete keywords from the given language
        escapeinside={{\%*}{*}},                 % if you want to add LaTeX within your code
        firstnumber=1,                          % start line enumeration with line 1
        frame=tb,                               % adds a frame around the code
        keepspaces=true,                        % keeps spaces in text
        keywordstyle=\color{pine},              % keyword style
        language=C++,                           % language of the code
        % morekeywords={*,...},                 % add more keywords to the set
        numbers=left,                           % position of line-numbers; possible values are (none, left, right)
        numbersep=15pt,                         % distance between line-numbers and the code
        numberstyle=\scriptsize\ttfamily\color{subtle}, % style used for line-numbers
        % rulecolor=\color{black},              % if not set, the frame-color may be changed on line-breaks within not-black text (e.g. comments (green here))
        showspaces=false,                       % show spaces everywhere
        showstringspaces=false,                 % underline spaces within strings only
        showtabs=false,                         % show tabs within strings
        stepnumber=1,                           % step between two line-numbers
        stringstyle=\color{red},               % string literal style
        tabsize=4,	                          % default tabsize
        % title=\lstname                        % show the filename
}

\begin{document}
    \subsection{Fonctionnement du système}
        \subsubsection{Inscription}
        \addimg{Reg.eps}{width=\linewidth}{Register process}{regSeq}
        \subsubsection{Connexion}
        \addimg{Login.eps}{width=\linewidth}{Login process}{loginSeq}
        \subsubsection{Chat}
        \addimg{ChatSequence.eps}{width=\linewidth}{Chat process}{chatSeq}
        \subsubsection{Calcul du ELO}
        Pour comprendre comment fonctionnera le matchmaking, nous devons comprendre comment fonctionne le systeme de classement ELO, en une phrase:
        \textit{"Le classement Elo est un système d’évaluation comparatif du niveau de jeu des joueurs d’échecs, de go ou d’autres jeux en un contre un."}\footnote{Wikipedia, Classement ELO} \\
        L'ELO est different de modalité en modalité. \\
        Pour le calculer, nous n'allons pas clairement inventer un nouveau algorithme, mais nous nous baserons sur ce qui a été déjà proposé\footnote{\href{https://metinmediamath.wordpress.com/2013/11/27/how-to-calculate-the-elo-rating-including-example/}{Metin's Media and Math, How To Calculate the Elo-Rating }}
        Bien que le jeu Quoridor peut etre joué par quatre joueur, nous allons adapter le calcul pour que le calcul reste équilibré. \\
        Supposons alors une prèmiere sitiuation pour une partie 1v1, avec deux joueurs 1 et 2 de ELO respectifs: 
    \begin{center}
        r(1) = 1500 et r(2) = 1753
    \end{center}
    Nous allons devoir calculer le "transformed score" R(n) comme suit:
    \begin{equation}
        \begin{split}
            & R(1) = 10^{\frac{r(1)}{400}} = 5623,413 \\
            & R(2) = 10^{\frac{r(2)}{400}} = 24126,815 \\
        \end{split}
    \end{equation}

    Calculons ensuite le "expected score" E(n):
    \begin{equation}
        \begin{split}
            & E(1) = R(1) / ( R(1) + R(2)) = 5623,413 / (5623,413 + 24126,815) = 0,189 \\
            & E(2) = R(2) / ( R(1) + R(2)) = 24126,815 / (5623,413 + 24126,815) = 0,8109 \\
        \end{split}
    \end{equation}

    Definissons le "actual score" pour les 2 joueurs:
    \begin{center}
        S(n) = 1 s'il gagne / 0 s'il perd
    \end{center}
    
    Calculons l'ELO final r'(n):
    Pour cela nous devons definir K: \\
    \textit{"This is called the K-factor and basically a measure of how strong a match will impact the players’ ratings.
    If you set K too low the ratings will hardly be impacted by the matches and very stable ratings (too stable) will occur.
    On the other hand, if you set it too high, the ratings will fluctuate wildly according to the current performance.
     Different organizations use different K-factors, there’s no universally accepted value. In chess the ICC uses a value of K = 32".}\footnote{\footnote{\href{https://metinmediamath.wordpress.com/2013/11/27/how-to-calculate-the-elo-rating-including-example/}{Metin's Media and Math, How To Calculate the Elo-Rating }}} \\
    Ici, nous supposons que le joueur 1 ait gagné
    \begin{equation}
        \begin{split}
           & r'(1) = r(1) + K \cdot (S(1) - E(1)) = 1500 + 32 \cdot (1 - 0,189) = 1525,952 \\
           & r'(2) = r(2) + K \cdot (S(2) - E(2)) = 1753 + 32 \cdot (0-0,8109) = 1727,0512 \\
        \end{split}
    \end{equation}

    Les ELO finaux des deux joueurs seront : \\

        r(1) = 1526, +26 \ r(2) = 1727, -26
    
    Supposons maintenant la partie se dèroule entre 4 joueurs, 1, 2, 3 et 4:
    \begin{center}
        r(1) = 1700, r(2) = 1900, r(3) = 1500, r(4) = 1600
    \end{center}
    Comme avant, calculons les "Transformed scores":
        \begin{equation}
            \begin{split}
                & R(1) = 10^{\frac{1700}{400}} = 17782,794 \\
                & R(2) = 10^{\frac{1900}{400}} = 56243,1325 \\
                & R(3) = 10^{\frac{1500}{400}} = 5623, 413  \\
                & R(4) = 10^{\frac{1600}{400}} = 10000 \\
            \end{split}
        \end{equation}
        Pour les "expected scores" E(n) pour chaque joueur, nous considererons la moyenne ponderée des "Transformed scores" R(m) des autres joueurs. 
        Nous voulons que la moyenne soit ponderée pour tenir compte de la difference entre les ELO des joueurs, une disparité plus grande affectera 
        de manière differente l'ELO final dans les cas d'une victoire ou d'une dèfaite.  \\
        Nous procéderons comme suit: \\
        Pour deux joueurs n et m: $\varepsilon = $ r(n) - r(m)  \\
        Nous considerons $< -100 \text{ et } > 100$ comme un ècart significatif
            $$ \omega_{n, m} =
            \begin{cases}
                1 + \varepsilon \cdot 10^{-3}, & \text{si $\varepsilon \le -100$} \\
                1 - \varepsilon \cdot 10^{-3}, & \text{si $\varepsilon \ge 100$} \\
                1, & \text{si $-100 < \varepsilon < 100$ }\\
            \end{cases} $$
        
        E(1) sera alors, avec J le nombre de joueurs adversaires (3): 
        \begin{equation}
            \begin{split}
                & E(1) = R(1) / [R(1) + (\dfrac{R(2) \cdot \omega_{1, 2} + R(3) \cdot \omega_{1, 3} + R(4) \cdot \omega_{1, 4}}{J})] = 0,394 \\
            \end{split}
        \end{equation}
        E(2), E(3) et E(4) se calculeront comme suit:
        \begin{equation}
            \begin{split}
                & E(2) = R(2) / [R(2) + (\dfrac{R(1) \cdot \omega_{2, 1} + R(3) \cdot \omega_{2, 3} + R(4) \cdot \omega_{2,4} }{J})] = 0,872 \\
                & E(3) = R(3) / [R(3) + (\dfrac{R(1) \cdot \omega_{3, 1} + R(2) \cdot \omega_{3, 2} + R(4) \cdot \omega_{3,4} }{J})] = 0,131 \\
                & E(4) = R(4) / [R(4) + (\dfrac{R(1) \cdot \omega_{4, 1} + R(2) \cdot \omega_{4, 2} + R(3) \cdot \omega_{4,3} }{J})] = 0,232 \\
            \end{split}
        \end{equation}
        S(n) sera defini comme avant : 1 si le joueur a gagné, 0 s'il a perdu. \\
        
        Le calcul du ELO final r'(n) s'effectue de la meme manière:
        \begin{equation}
            \begin{split}
                & r'(1) = r(1) + K \cdot (S(1) - E(1)) = 1719 \\
                & r'(2) = r(2) + K \cdot (S(2) - E(2)) = 1872 \\
                & r'(3) = r(3) + K \cdot (S(3) - E(3)) = 1495 \\
                & r'(4) = r(4) + K \cdot (S(4) - E(4)) = 1592 \\
            \end{split}
        \end{equation}

        L'ELO final pour chaque joueur correspondra alors à r'(n). \\
        Nous pouvons constater que les ELO finaux sont raisonnables si mis en rapport avec la disparité de ELO initiale. Les joueurs 3 et
        4 ne voient pas leur ELO diminuer beaucoup, par contre le joueur 2 qui avait l'ELO le plus haut, constate une diminution de 28.



\end{document}