\section{Base de donnée}
Afin de réaliser ce projet, il est primordial pour le serveur de stocker les données continuellement et non pas seulement pendant l'exécution du programme. Pour ce faire, il est nécessaire d'utiliser une base de donnée. De ce fait, plusieurs choix s'offrent à nous.
Nous allons évaluer ces choix potentiels en prenant en compte leur utilité dans le cadre de ce projet.

\subsection{SQL}
Le \textit{relational data model} que SQL apporte, domine le type de base de données sur l'internet moderne. Ce modèle de stockage peut s'avérer utile dans le cadre de ce projet.

\subsubsection{SQLite}
SQLite permet d'implémenter rapidement et facilement ce modèle. En revanche, SQLite n'est pas idéal pour la gestion d'utilisateur et la concurrence puisque seulement un seul processus peut modifier la base de donnée à la fois.

\subsubsection{MySQL}
MySQL est l'implémentation de ce modèle la plus populaire. MySQL semble être un bon choix pour ce projet de part pour sa fiabilité mais aussi pour sa popularité qui entraînera donc une facilité lors de recherches à son sujet.

\subsubsection{PostgreSQL}
PostgreSQL est un choix solide puisqu'il a la plupart des mêmes avantages de MySQL mais a en plus la capacité de stocker plus de types de données nativement, tel que JSON, ce qui pourraient s'avérer très utile. Un inconvénient de PostgreSQL est que la performance au niveau de la mémoire n'est pas idéale mais cela ne devrait pas poser de problème vu que nous n'allons à priori pas gérer plusieurs milliers de joueurs à la fois.

\subsection{MongoDB et autre base de données NoSQL}
MongoDB est un modèle de base de donnée plus moderne se basant sur le NoSQL. MongoDB peut s'avérer être un bon choix grâce à sa grande flexibilité et sa rapidité. MongoDB est néanmoins moins populaire que SQL mais semble être un choix idéal pour ce projet.

\subsection{Fichier text}
Une autre méthode de stockage de données serait un simple fichier texte. Cela n'est pas idéal dû principalement au manque de structure que cela apporte.

\subsection{Décision}
La décision concernant la base de donnée n'a pas encore été faite mais à premier abord MongoDB et PostgreSQL semble être des choix idéals pour accomplir ce projet. Une base de données se basant sur l'un ou l'autre nécessitera néanmoins une grande part de travail pour leur intégration au serveur.
Il s'avèrera sûrement utile de faire tourner la base de donnée et le serveur dans des containers à l'aide de Docker par exemple afin de simplifier la reproduction du serveur sur d'autres machines.
